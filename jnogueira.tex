%%%%%%%%%%%%%%%%%
% This is an example CV created using altacv.cls (v1.1.3, 30 April 2017) written by
% LianTze Lim (liantze@gmail.com), based on the
% Cv created by BusinessInsider at http://www.businessinsider.my/a-sample-resume-for-marissa-mayer-2016-7/?r=US&IR=T
%
%% It may be distributed and/or modified under the
%% conditions of the LaTeX Project Public License, either version 1.3
%% of this license or (at your option) any later version.
%% The latest version of this license is in
%%    http://www.latex-project.org/lppl.txt
%% and version 1.3 or later is part of all distributions of LaTeX
%% version 2003/12/01 or later.
%%%%%%%%%%%%%%%%

%% If you want to use \orcid or the
%% academicons icons, add "academicons"
%% to the \documentclass options.
%% Then compile with XeLaTeX or LuaLaTeX.
% \documentclass[10pt,a4paper,academicons]{altacv}

%% Use the "normalphoto" option if you want a normal photo instead of cropped to a circle
% \documentclass[10pt,a4paper,normalphoto]{altacv}

\documentclass[10pt,a4paper]{altacv}

%% AltaCV uses the fontawesome and academicon fonts
%% and packages.
%% See texdoc.net/pkg/fontawecome and http://texdoc.net/pkg/academicons for full list of symbols.
%% When using the "academicons" option,
%% Compile with LuaLaTeX for best results. If you
%% want to use XeLaTeX, you may need to install
%% Academicons.ttf in your operating system's font %% folder.


% Change the page layout if you need to
\geometry{left=1cm,right=9cm,marginparwidth=6.8cm,marginparsep=1.2cm,top=1cm,bottom=1cm}

% Change the font if you want to.

% If using pdflatex:
\usepackage[utf8]{inputenc}
\usepackage[T1]{fontenc}
\usepackage[default]{lato}
\usepackage{tabularx}
\usepackage{lmodern,textcomp}
\usepackage[hidelinks]{hyperref}
\usepackage[sfdefault]{roboto}
\usepackage{academicons}
\usepackage{setspace}

% If using xelatex or lualatex:
% \setmainfont{Lato}

% Change the colours if you want to
\definecolor{VividPurple}{HTML}{0E4D92}
\definecolor{SlateGrey}{HTML}{4682B4}
\definecolor{LightGrey}{HTML}{666666}
\colorlet{heading}{VividPurple}
\colorlet{accent}{VividPurple}
\colorlet{emphasis}{SlateGrey}
\colorlet{body}{LightGrey}

% Change the bullets for itemize and rating marker
% for \cvskill if you want to
\renewcommand{\itemmarker}{{\small\textbullet}}
\renewcommand{\ratingmarker}{\faCircle}

\begin{document}
\name{João Nogueira}
\tagline{}
% Cropped to square from https://en.wikipedia.org/wiki/Marissa_Mayer#/media/File:Marissa_Mayer_May_2014_(cropped).jpg, CC-BY 2.0
\photo{2.5cm}{foto_square.png}
\personalinfo{%
	\begin{spacing}{1.5}
		\birthday{30/03/1987}
		\location{\href{https://www.google.com/maps/@?api=1\&map\_action=map\&center=40.6408378\%2C-8.641316\&zoom=10}{Aveiro, Portugal}}
		\email{\href{mailto:joaopbnogueira@gmail.com}{joaopbnogueira@gmail.com}}  
		\phone{\href{tel:+351962924365}{+351 962924365}}
		\skype{\href{skype:joao.p.nogueira}{joao.p.nogueira}}
		\linkedin{\href{https://www.linkedin.com/in/joaopbnogueira/}{linkedin.com/in/joaopbnogueira}}
		\orcid{\href{https://orcid.org/0000-0002-5748-833X}{orcid.org/0000-0002-5748-833X}}
	%  \email{--------------- }  
	%  \phone{---------------}
	%  \linkedin{linkedin.com/in/joaopbnogueira}
	%  \skype{---------------}
	\end{spacing}
}

%% Make the header extend all the way to the right, if you want.
\begin{fullwidth}
\makecvheader
\end{fullwidth}

%% Provide the file name containing the sidebar contents as an optional parameter to \cvsection.
%% You can always just use \marginpar{...} if you do
%% not need to align the top of the contents to any
%% \cvsection title in the "main" bar.
\cvsection[p1sidebar]{Key Expertise}

\begin{tabularx}{\linewidth}{X X}
	• Large-scale systems         & • International projects     \\
	• Multiple program. languages & • Agile Scrum teams          \\
	• Clean code, tests, CI/CD    & • Autonomous and self-driven
\end{tabularx}

\cvsection{Industry Experience}

\cvevent{Senior Full Stack Developer}{\href{https://www.bosch.com/}{Bosch Thermotechnology}}{}{Aug 2017 -- Present}{Aveiro, Portugal}
\begin{itemize}
	\item \textbf{Deployment and operation of Azure cloud services} in a \textbf{Continuous Integration / Continuous Deployment} fashion
	\item Cross-functional development, in \textbf{backend and frontend code},  working in \textbf{Agile Scrum} supporting the \textbf{international team} on everything that might be needed to push the project forward
\end{itemize}

\divider

\cvevent{Software Developer (Research \& Development)}{\href{http://www.alticelabs.com/}{Altice Labs}}{Formerly Portugal Telecom Inovação}{Jun 2012 -- Aug 2017}{Aveiro, Portugal}
\begin{itemize}
	\item Major contributions to \href{https://www.meo.pt/tv}{MEO IPTV} \textbf{large scale production services}, defining their \textbf{architecture and key algorithms} to ensure \textbf{performance and scalability} goals -- e.g. lead developer \& architect of MEO Gravações Automáticas, with more than 1 million users every month
	\item Development and curation of reusable libraries / components in order to speedup new applications development, reduce production bugs, and \textbf{promote clean code and good programming practices}
	\item \textbf{Planning, management, and coordination responsibilities in large national \& international projects} -- e.g. \href{http://www.alticelabs.com/gapott/}{Celtic-Plus NOTTS} (8.5 Million € budget on 6 different countries) and \href{http://www.alticelabs.com/site/ultratv/}{Ultra TV}
\end{itemize}

\divider

\cvevent{Software Developer (Trainee Internship)}{\href{http://www.alticelabs.com/}{Altice Labs}}{Formerly Portugal Telecom Inovação}{Aug 2010 -- Sep 2011}{Aveiro, Portugal}

\begin{itemize}
\item Successfully designed \& developed a Network / Cloud DVR product, \textbf{working autonomously} and proposing novel solutions to its significant technical challenges on client and server implementations
\end{itemize}

\clearpage

\cvsection[p2sidebar]{Tech Stack}

\cvtag{C\#} \cvtag{.Net (Core)} \cvtag{MVVM} \cvtag{MVC} 
\cvtag{WebApi} \cvtag{REST} \cvtag{IIS} \cvtag{Swagger}
\cvtag{SQL}\cvtag{OData} \cvtag{Azure} \cvtag{Autofac} \cvtag{Moq} \cvtag{Serilog}
\cvtag{Entity Framework}  \cvtag{Mediaroom} \cvtag{IPTV} \cvtag{WCF} \cvtag{OTT} \cvtag{Smooth Streaming} \cvtag{HLS} \cvtag{MPEG-DASH} \cvtag{Git} \cvtag{Svn} \cvtag{JIRA}

\divider
Also worked with: \\
\cvtag{React} \cvtag{Angular} \cvtag{jQuery}  \cvtag{Javascript} \cvtag{C++} \cvtag{C} \cvtag{Java}
\cvtag{R} \cvtag{Python} \cvtag{C++} \cvtag{C} \cvtag{Java} \cvtag{Machine Learning} \cvtag{Big Data} \cvtag{RStudio}
\cvtag{CDNs} \cvtag{Apache Traffic Server} \cvtag{Nginx} \cvtag{Varnish}

%
%\nocite{*}
%
%\printbibliography[heading=pubtype,title={\printinfo{\faBook}{Books}},type=book]
%
%\divider
%
%\printbibliography[heading=pubtype,title={\printinfo{\faFileTextO}{Journal Articles}}, type=article]
%
%\divider
%
%\printbibliography[heading=pubtype,title={\printinfo{\faGroup}{Conference Proceedings}},type=inproceedings]

%% If the NEXT page doesn't start with a \cvsection but you'd
%% still like to add a sidebar, then use this command on THIS
%% page to add it. The optional argument lets you pull up the
%% sidebar a bit so that it looks aligned with the top of the
%% main column.
% \addnextpagesidebar[-1ex]{page3sidebar}


\end{document}
